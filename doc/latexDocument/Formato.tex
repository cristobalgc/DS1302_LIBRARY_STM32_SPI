%\usepackage{blindtext}
\usepackage[utf8]{inputenc}
\usepackage[spanish,es-lcroman,es-nosectiondot,es-tabla]{babel}
%------------------------------------------------------
\usepackage{array,amssymb,amsthm,amsmath,amstext}
\usepackage{afterpage}% Necesario para introdicur páginas A3
\usepackage[font=scriptsize,bf]{caption}% Formato del caption
\usepackage{colortbl}% permite colorear tablas
\usepackage{booktabs} % To thicken table lines
\usepackage{longtable} % para tablas largas
\usepackage{eurosym}
\usepackage{wrapfig}
\usepackage{emptypage}% evita la numeración de las páginas en blanco
\usepackage{fancyhdr,float}
\usepackage{graphicx}
\usepackage{listings} % premite la introducción de códigos vhdl
\usepackage{lscape}% Necesario para páginas apaisadas
\usepackage{mcode,multirow}
\usepackage{pdfpages}
\usepackage[paper=A4,pagesize]{typearea}%necesario para introducir páginas A3
\usepackage{setspace,subfigure}
\usepackage{titlesec}
\usepackage{xcolor}
\usepackage[3D]{movie15}
\usepackage{pdflscape}
\usepackage[colorlinks=true,linkcolor=udc,citecolor=udc]{hyperref}

% Fuente ----------------------------------------------------------
\usepackage{helvet}
\renewcommand*\familydefault{\sfdefault} 
%------------------------------------------------------------------
% Comandos personales
\definecolor{udc}{rgb}{0.81,0,0.49} % color de la UDC
\spanishdecimal{.} % escribe . en lugar de , para separar enteros de decimales
%\newcommand\crule[3][black]{\textcolor{#1}{\rule{#2}{#3}}}
\allowdisplaybreaks
%\renewcommand{\refname}{Referencias}
\newcommand{\IE}[3]{${}^{\scriptstyle^{#2}}_{\scriptstyle{\raisebox{-4pt}{}^{#3}}}{\text{#1}}$} % Elementos químicos

% Formato hoja----------------------------------------------
\renewcommand{\baselinestretch}{1.3}% interlineado
\headsep 15.5mm      \topmargin -2cm    \textheight 24.5cm     \textwidth 17cm    \oddsidemargin 0.04 cm   \evensidemargin -1.04cm
\footnotesep=20pt   \footskip=40pt
% Formato de las cabaceras de página -----------------------------------------------------------------
\usepackage{fancyhdr}
\pagestyle{fancy}
\fancyhf{} % borrar todos los ajustes
\fancyfoot[C]{\textbf{\scriptsize\thepage}}
% Modifica el ancho de las líneas de cabecera y pie
\renewcommand{\headrulewidth}{0pt}
%\renewcommand{\footrulewidth}{0.4pt}
\renewcommand{\footrule}{\hrule height 0.4pt \vspace{4mm}}
%%%%%%%%%%%%%%%%%%%%%%%%%%%%%%%%%%%%%%%%%%%%%%%%%%%%%%%%%%%%%%%%%%%%%%%%%%%%%%%%%%%%%%%%%%%%%%%%%%%%%%%%%%%%%%%%%%%%%%%%%%%%%%%%%%%
\usepackage{float}
\newfloat{Plano}{p}{pln}%[chapter]
\captionsetup[Plano]{labelformat=empty,labelsep=none,position=below}
%%%%%%%%%%%%%%%%%%%%%%%%%%%%%%%%%%%%%%%%%%%%%%%%%%%%%%%%%%%%%%%%%%%%%%%%%%%%%%%%%%%%%%%%%%%%%%%%%%%%%%%%%%%%%%%%%%%%%%%%%%%%%%%%%%%%
%--------------------------------------------------------------------
\setcounter{lofdepth}{2}% Introduce las subfiguras en lof
\allowdisplaybreaks
%%%%%%%%%%%%%%%%%%%%%%%%%%%%%%%%%%%%%%%%%%%%%%%%%%%%%%%%%%%%%%%%%%%%%%%%%%%%%%%%%%%%%%%%%%%
% Incluye la bibliografía como sección
%\makeatletter
%\renewenvironment{thebibliography}[1]
     %{\section{\refname}% esta línea cambia la bibliografía a la categoría sección
      %\@mkboth{\MakeUppercase\bibname}{\MakeUppercase\bibname}
      %\list{\@biblabel{\@arabic\c@enumiv}}%
           %{\settowidth\labelwidth{\@biblabel{#1}}
            %\leftmargin\labelwidth
            %\advance\leftmargin\labelsep
            %\@openbib@code
            %\usecounter{enumiv}
            %\let\p@enumiv\@empty
            %\renewcommand\theenumiv{\@arabic\c@enumiv}}
      %\sloppy
      %\clubpenalty4000
      %\@clubpenalty \clubpenalty
      %\widowpenalty4000%
      %\sfcode`\.\@m}
     %{\def\@noitemerr
       %{\@latex@warning{Empty `thebibliography' environment}}
      %\endlist}
%\makeatother
%%%%%%%%%%%%%%%%%%%%%%%%%%%%%%%%%%%%%%%%%%%%%%%%%%%%%%%%%%%%%%%%%%%%%%%%%%%%%%%%%%%%%%%%%%%%%%%%%%%%%%%%%%%%%%%%%%%%%%%%%%%%%%%%%%%%%%
\renewcommand\thesection{\arabic{section}}

% Se cambio el nombre del caption para los códigos
\renewcommand\lstlistingname{Código}
%%%%%%%%%%%%%%%%%%%%%%%%%%%%%%%%%%%%%%%%%%%%%%%%%%%%%%%%%%%%%%%%%%%%%%%%%%%%%%%%%%%%%%%%%%%%%%%%%%%%%%%%%%%%%%%%%%%%%%%%%%%%%%%%%%
% Incluye la bibliografía como subsección
\makeatletter
\renewenvironment{thebibliography}[1]
     {\subsection{\bibname}% esta línea cambia la bibliografía a la categoría sección
      \@mkboth{\MakeUppercase\bibname}{\MakeUppercase\bibname}
      \list{\@biblabel{\@arabic\c@enumiv}}%
           {\settowidth\labelwidth{\@biblabel{#1}}
            \leftmargin\labelwidth
            \advance\leftmargin\labelsep
            \@openbib@code
            \usecounter{enumiv}
            \let\p@enumiv\@empty
            \renewcommand\theenumiv{\@arabic\c@enumiv}}
      \sloppy
      \clubpenalty4000
      \@clubpenalty \clubpenalty
      \widowpenalty4000%
      \sfcode`\.\@m}
     {\def\@noitemerr
       {\@latex@warning{Empty `thebibliography' environment}}
      \endlist}
\makeatother
% Contador para todas las referencias
\newcounter{contador}
%%%%%%%%%%%%%%%%%%%%%%%%%%%%%%%%%%%%%%%%%%%%%%%%%%%%
\lstset{literate=%
         {á}{{\'a}}1 {é}{{\'e}}1 {í}{{\'i}}1 {ó}{{\'o}}1  {ú}{{\'u}}1 {ñ}{{\~n}}1
         {Á}{{\'A}}1 {É}{{\'E}}1 {Í}{{\'I}}1 {Ó}{{\'O}}1 {Ú}{{\'U}}1}